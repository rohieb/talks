\documentclass{beamer}
%\documentclass[handout]{beamer}

\usepackage[utf8x]{inputenc}
\usepackage{color}
%\usepackage{multirow}
\usepackage{graphicx}
\usepackage{listings}

\lstset{%
  language=C++,
  basicstyle=\scriptsize\ttfamily
}

\usetheme{Boadilla}
\usefonttheme{professionalfonts}
\useoutertheme[subsection=false,footline=empty]{miniframes}
\useinnertheme{circles}
\setbeamertemplate{footline}[frame number]

\author{Roland Hieber}
\title{Copy-on-write und Implicit Sharing in Qt}
\institute{Stratum~0~e.~V.}
\date{14.~Dezember~2012}

\begin{document}
\begin{frame}
  \titlepage
  \thispagestyle{empty}
\end{frame}

\section{Motivation}
\begin{frame}{Motivation}
  \lstinputlisting{motivation.cpp}
\end{frame}
\begin{frame}{Motivation}
  \lstinputlisting[basicstyle=\tiny\ttfamily]{motivation.cpp}
  \begin{itemize}
    \item drei Kopien eines \texttt{Node}-Objekt
    \item braucht jedesmal Speicherplatz
    \begin{itemize}
      \item obwohl der Inhalt identisch ist
    \end{itemize}
    \item Lösungsmöglichkeiten: (Shared) Pointer, Copy-on-write
  \end{itemize}
\end{frame}

\section{Copy-on-write}
\begin{frame}{Copy-on-write}
  Methodik:
  \begin{itemize}
    \item Kopien teilen sich vorerst einen Speicherbereich 
      \emph{(shallow copy)}
    \item Kopien werden erst wirklich kopiert, wenn sie geändert werden
      \emph{(deep copy)}
  \end{itemize}
\end{frame}

\begin{frame}{Copy-on-write}
  Objekt kopieren:
    \begin{itemize}
      \item vorerst auf Speicherbereich des alten Objekts verweisen
      \item gemeinsamen Referenzzähler inkrementieren
    \end{itemize}
\end{frame}

\begin{frame}{Copy-on-write}
  Objekt ändern:
    \begin{itemize}
      \item Falls Referenzzähler gleich 1:
        \begin{itemize}
          \item \emph{nur ein Objekt referenziert gemeinsamen Speicher}
          \item Änderungen wie gewohnt vornehmen
        \end{itemize}
      \item Falls Referenzzähler größer als 1:
        \begin{itemize}
          \item \emph{mehr als ein Objekt referenziert gemeinsamen Speicher}
          \item alten Referenzzähler dekrementieren
          \item neuen Speicher reservieren
          \item alten Speicherbereich kopieren
          \item neuen Referenzzähler auf 1 setzen
          \item Änderungen wie gewohnt in neuem Speicherbereich vornehmen
        \end{itemize}
    \end{itemize}
\end{frame}

\begin{frame}{Copy-on-write}
  Objekt löschen:
    \begin{itemize}
      \item gemeinsamen Referenzzähler dekrementieren
      \item Falls Referenzzähler 0:
        \begin{itemize}
          \item \emph{kein Objekt referenziert mehr den gemeinsamen Speicher}
          \item Speicherbereich freigeben
        \end{itemize}
    \end{itemize}
\end{frame}

\section{Umsetzung in Q}
\begin{frame}{Umsetzung in Qt}
  Implementierung als Flyweight Pattern:
  \begin{itemize}
    \item \texttt{NodeData} speichert nur Daten
      \begin{itemize}
        \item abgeleitet von \texttt{QSharedData}
        \item threadsichere Referenzzählung
      \end{itemize}
    \pause
    \item \texttt{Node} als Wrapper für Zugriff von außen
      \begin{itemize}
        \item enthält Member vom Typ \texttt{QSharedDataPointer<NodeData>}
          \begin{itemize}
            \item Shared Pointer auf QSharedData-Objekte
            \item \texttt{detach()} erstellt deep copies
            \item Überladung von \texttt{operator->()}
              ruft bei Schreibzugriff \texttt{detach()} auf
          \end{itemize}
        \item Copy-Konstruktor kopiert nur Referenz auf
          \texttt{QSharedDataPointer<NodeData>}!
      \end{itemize}
  \end{itemize}
\end{frame}

\begin{frame}{Beispiel}
  nodedata.h
  \lstinputlisting{nodedata.h}
\end{frame}

\begin{frame}{Beispiel}
  node.cpp
  \lstinputlisting{node.cpp}
\end{frame}

\section{Quellen}
\begin{frame}
  \centering
  \vfill
  Vielen Dank für die Aufmerksamkeit!
  \vfill
  {\footnotesize Diese Vortragsfolien sind lizenziert unter CC-BY-SA 3.0.}
  \vfill
  \begin{thebibliography}{9}
    \bibitem{qt-ic}{Qt Documentation: Implicit Sharing.
      \url{https://qt-project.org/doc/qt-4.8/implicit-sharing.html}}
    \bibitem{qt-qsd}{Qt Documentation: QSharedData.
      \url{https://qt-project.org/doc/qt-4.8/qshareddata.html}}
    \bibitem{qt-qsdp}{Qt Documentation: QSharedDataPointer.
      \url{https://qt-project.org/doc/qt-4.8/qshareddatapointer.html}}
    \bibitem{wiki}{Wikipedia: \emph{Flyweight Pattern}.
      \url{https://en.wikipedia.org/w/index.php?oldid=526237500}}
  \end{thebibliography}
\end{frame}
\end{document}
% vim: set sw=2 ts=2 expandtab :
