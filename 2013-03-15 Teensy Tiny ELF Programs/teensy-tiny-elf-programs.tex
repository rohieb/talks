\documentclass{beamer}
\usepackage[utf8x]{inputenc}
\usepackage{color}
\usepackage{listings}
\usepackage{graphicx} % needed for beamer

\lstset{%
  language=C,
  basicstyle=\ttfamily
}

\usetheme{Boadilla}
\usefonttheme{professionalfonts}
\useoutertheme[subsection=false,footline=empty]{miniframes}
\useinnertheme{circles}
\setbeamertemplate{footline}[frame number]

\author{Roland Hieber}
\title{Teensy Tiny ELF Programs}
\subtitle{inspired by Brian Raiter \\}
\institute{Stratum~0~e.\,V.}
\date{March~15, 2013}

\begin{document}
\begin{frame}
	\maketitle
\end{frame}


\begin{frame}{Hello World}
\lstinputlisting{hello.c}
\pause
\begin{itemize}
	\item Okay. Maybe don't print anything.
	\pause
	\item Oh okay, strip the executable.
\end{itemize}
\end{frame}

\begin{frame}[fragile]
\frametitle{Next step: Assembler}
\begin{lstlisting}
	; tiny.asm
	BITS 32
	GLOBAL main
	SECTION .text
	main:
	  mov     eax, 42
	  ret
\end{lstlisting}
\pause
\begin{lstlisting}
$ nasm -f elf tiny.asm
$ gcc -Wall -s tiny.o
$ ./a.out ; echo $?
42
$ wc -c a.out
   2604 a.out
\end{lstlisting}
\end{frame}

\begin{frame}[fragile]
\frametitle{Deeper into the Rabbit Hole: libc}
\begin{lstlisting}
	; tiny.asm
	BITS 32
	EXTERN _exit
	GLOBAL _start
	SECTION .text
	_start:
	  push    dword 42
	  call    _exit
\end{lstlisting}
\pause
\begin{lstlisting}
$ nasm -f elf tiny.asm
$ gcc -Wall -s -nostartfiles tiny.o
$ ./a.out ; echo $?
42
$ wc -c a.out
   1340 a.out
\end{lstlisting}
\end{frame}

\begin{frame}[fragile]
\frametitle{But\ldots do we even need libc?}
\begin{lstlisting}
; tiny.asm
BITS 32
GLOBAL _start
SECTION .text
_start:
  mov     eax, 1  ; "exit" syscall, see unistd.h
  mov     ebx, 42  
  int     0x80
\end{lstlisting}
\pause
\begin{lstlisting}
$ nasm -f elf tiny.asm
$ gcc -Wall -s -nostdlib tiny.o
$ ./a.out ; echo $?
42
$ wc -c a.out
  372 a.out
\end{lstlisting}
\end{frame}

\begin{frame}[fragile]
\frametitle{Okay, what does our executable contain?}
\begin{lstlisting}[basicstyle=\scriptsize\ttfamily]
$ objdump -x a.out | less
[...]
  Sections:
  Idx Name        Size      VMA       LMA       File off  Algn
    0 .text       00000007  08048080  08048080  00000080  2**4
                  CONTENTS, ALLOC, LOAD, READONLY, CODE
    1 .comment    0000001c  00000000  00000000  00000087  2**0
                  CONTENTS, READONLY
[...]

$ hexdump a.out
00000080: 31C0 40B3 2ACD 8000 5468 6520 4E65 7477  1.@.*...The Netw
00000090: 6964 6520 4173 7365 6D62 6C65 7220 302E  ide Assembler 0.
000000A0: 3938 0000 2E73 796D 7461 6200 2E73 7472  98...symtab..str
\end{lstlisting}
\end{frame}

\begin{frame}[fragile]
\frametitle{Hmm. Let's write ELF directly.}
\begin{lstlisting}[basicstyle=\scriptsize\ttfamily]
BITS 32
       org  0x08048000
ehdr:                               ; Elf32_Ehdr
        db  0x7F, "ELF", 1, 1, 1, 0 ;  e_ident
times 8 db  0
        dw  2                       ;  e_type
        dw  3                       ;  e_machine
        dd  1                       ;  e_version
        dd  _start                  ;  e_entry
        dd  phdr - $$               ;  e_phoff
        dd  0                       ;  e_shoff
        dd  0                       ;  e_flags
        dw  ehdrsize                ;  e_ehsize
        dw  phdrsize                ;  e_phentsize
        dw  1                       ;  e_phnum
        dw  0                       ;  e_shentsize
        dw  0                       ;  e_shnum
        dw  0                       ;  e_shstrndx
ehdrsize  equ  $ - ehdr
\end{lstlisting}
\end{frame}

\begin{frame}[fragile]
\frametitle{Hmm. Let's write ELF directly.}
\begin{lstlisting}[basicstyle=\scriptsize\ttfamily]
phdr:                               ; Elf32_Phdr
        dd  1                       ;  p_type
        dd  0                       ;  p_offset
        dd  $$                      ;  p_vaddr
        dd  $$                      ;  p_paddr
        dd  filesize                ;  p_filesz
        dd  filesize                ;  p_memsz
        dd  5                       ;  p_flags
        dd  0x1000                  ;  p_align
phdrsize  equ $ - phdr

_start:
; your program here
filesize      equ     $ - $$
\end{lstlisting}
\end{frame}

\begin{frame}[fragile]
\frametitle{Hmm. Let's write ELF directly.}
\begin{lstlisting}
$ nasm -f bin -o a.out tiny.asm
$ chmod +x a.out
$ ./a.out ; echo $?
42
$ wc -c a.out
     91 a.out
\end{lstlisting}
\end{frame}

\begin{frame}[fragile]
\frametitle{But\ldots the spec doesn't forbid overlapping headers\ldots}
\begin{lstlisting}[basicstyle=\tiny\ttfamily]
; tiny.asm
BITS 32
         org     0x00200000
         db      0x7F, "ELF"     ; e_ident
         db      1, 1, 1, 0, 0
_start:  mov     bl, 42
         xor     eax, eax
         inc     eax
         int     0x80
         dw      2               ; e_type
         dw      3               ; e_machine
         dd      1               ; e_version
         dd      _start          ; e_entry
         dd      phdr - $$       ; e_phoff
phdr:    dd      1               ; e_shoff       ; p_type
         dd      0               ; e_flags       ; p_offset
         dd      $$              ; e_ehsize      ; p_vaddr
                                 ; e_phentsize
         dw      1               ; e_phnum       ; p_paddr
         dw      0               ; e_shentsize
         dd      filesize        ; e_shnum       ; p_filesz
                                 ; e_shstrndx
         dd      filesize                        ; p_memsz
         dd      5                               ; p_flags
         dd      0x1000                          ; p_align
filesize equ     $ - $$
\end{lstlisting}
\end{frame}

\begin{frame}[fragile]
\frametitle{But\ldots the spec doesn't forbid overlapping headers\ldots}
\ldots some other dirty hacks which probably only work for Linux\ldots
\begin{lstlisting}
$ nasm -f bin -o a.out tiny.asm
$ chmod +x a.out
$ ./a.out ; echo $?
42
$ wc -c a.out
     45 a.out
\end{lstlisting}

45 bytes for a valid Linux executable?! \textbackslash{}o/
(okay, ``valid''\ldots probably only works for libc)
\end{frame}

\begin{frame}{Sources}
\begin{thebibliography}{9}
	\bibitem{raiter}{Brian Raiter: A Whirlwind Tutorial on Creating Really Teensy
		ELF Executables for Linux.
		\url{http://www.muppetlabs.com/~breadbox/software/tiny/teensy.html}}
\end{thebibliography}
\end{frame}

\end{document}

