\documentclass{beamer}
\usepackage[utf8x]{inputenc}
\usepackage{color}
\usepackage{listings}
\usepackage{graphicx} % needed for beamer

\lstset{%
  language=C,
  basicstyle=\ttfamily,
	showspaces=false,
	showstringspaces=false,
	showtabs=false,
}

\usetheme{Boadilla}
\usefonttheme{professionalfonts}
\useoutertheme[subsection=false,footline=empty]{miniframes}
\useinnertheme{circles}
\setbeamertemplate{footline}[frame number]

\author{Roland Hieber}
\title{Teensy Tiny ELF Programs}
\subtitle{inspired by Brian Raiter}
\institute{Stratum~0~e.\,V.}
\date{March~15, 2013}

\begin{document}
\begin{frame}
	\maketitle
\end{frame}


\begin{frame}[fragile]{Hello World}
\begin{lstlisting}
#include <stdio.h>

int main(int argc, char** argv) {
	printf("Hello World!\n");
	return 42;
}
\end{lstlisting}
Well, how big can it be?
\pause

\begin{lstlisting}
$ gcc hello.c
$ wc -c a.out
4483 a.out
\end{lstlisting}

Oops.
\end{frame}

\begin{frame}[fragile]{Hello World}
\begin{itemize}
	\item Okay. Maybe don't print anything, just return a value.
	\begin{lstlisting}
		$ echo 'main(){return 42;}' | gcc -x c -
		$ wc -c a.out
		4359 a.out
	\end{lstlisting}
	\pause
	\item Oh okay, we forgot to optimize for size and strip the executable.
	\begin{lstlisting}
		$ echo 'main(){return 42;}' | gcc -x c -s -Os -
		$ wc -c a.out
		2756 a.out
	\end{lstlisting}
	\pause
	\item Minimal C program has still 2{.}7~KB. Meh.
\end{itemize}
\end{frame}

\begin{frame}[fragile]{Next step: Assembler}
\begin{lstlisting}
	; tiny.asm
	BITS 32
	GLOBAL main
	SECTION .text
	main:
	  mov     eax, 42
	  ret
\end{lstlisting}
\pause
\begin{lstlisting}
$ nasm -f elf tiny.asm
$ gcc -Wall -s tiny.o
$ ./a.out ; echo $?
42
$ wc -c a.out
   2604 a.out
\end{lstlisting}
\end{frame}

\begin{frame}[fragile]{Deeper into the Rabbit Hole: libc}
\begin{lstlisting}
	; tiny.asm
	BITS 32
	EXTERN _exit
	GLOBAL _start
	SECTION .text
	_start:
	  push    dword 42
	  call    _exit
\end{lstlisting}
\pause
\begin{lstlisting}
$ nasm -f elf tiny.asm
$ gcc -Wall -s -nostartfiles tiny.o
$ ./a.out ; echo $?
42
$ wc -c a.out
   1340 a.out
\end{lstlisting}
\end{frame}

\begin{frame}[fragile]
\frametitle{But\ldots do we even need libc?}
\begin{lstlisting}
; tiny.asm
BITS 32
GLOBAL _start
SECTION .text
_start:
  mov     eax, 1  ; "exit" syscall, see unistd.h
  mov     ebx, 42
  int     0x80
\end{lstlisting}
\pause
\begin{lstlisting}
$ nasm -f elf tiny.asm
$ gcc -Wall -s -nostdlib tiny.o
$ ./a.out ; echo $?
42
$ wc -c a.out
  372 a.out
\end{lstlisting}
\end{frame}

\begin{frame}[fragile]
\frametitle{Okay, what does our executable contain?}
\begin{lstlisting}[basicstyle=\scriptsize\ttfamily]
$ objdump -x a.out | less
[...]
  Sections:
  Idx Name        Size      VMA       LMA       File off  Algn
    0 .text       00000007  08048080  08048080  00000080  2**4
                  CONTENTS, ALLOC, LOAD, READONLY, CODE
    1 .comment    0000001c  00000000  00000000  00000087  2**0
                  CONTENTS, READONLY
[...]

$ hexdump a.out
[...]
00000080: 31C0 40B3 2ACD 8000 5468 6520 4E65 7477  1.@.*...The Netw
00000090: 6964 6520 4173 7365 6D62 6C65 7220 302E  ide Assembler 0.
000000A0: 3938 0000 2E73 796D 7461 6200 2E73 7472  98...symtab..str
[...]
\end{lstlisting}
NASM, that bitch.
\end{frame}

\begin{frame}[fragile]
\frametitle{Time for some black magic: let's write ELF directly.}
\begin{lstlisting}[basicstyle=\scriptsize\ttfamily]
; tiny.asm
BITS 32
       org  0x08048000
ehdr:                               ; Elf32_Ehdr, see <linux/elf.h>
        db  0x7F, "ELF", 1, 1, 1, 0 ;  e_ident
times 8 db  0                       ;    (padding)
        dw  2                       ;  e_type
        dw  3                       ;  e_machine
        dd  1                       ;  e_version
        dd  _start                  ;  e_entry
        dd  phdr - $$               ;  e_phoff
        dd  0                       ;  e_shoff
        dd  0                       ;  e_flags
        dw  ehdrsize                ;  e_ehsize
        dw  phdrsize                ;  e_phentsize
        dw  1                       ;  e_phnum
        dw  0                       ;  e_shentsize
        dw  0                       ;  e_shnum
        dw  0                       ;  e_shstrndx
ehdrsize  equ  $ - ehdr
\end{lstlisting}
\end{frame}

\begin{frame}[fragile]
\frametitle{Time for some black magic: let's write ELF directly.}
\begin{lstlisting}[basicstyle=\scriptsize\ttfamily]
; tiny.asm, cont'd
phdr:                               ; Elf32_Phdr
        dd  1                       ;  p_type
        dd  0                       ;  p_offset
        dd  $$                      ;  p_vaddr
        dd  $$                      ;  p_paddr
        dd  filesize                ;  p_filesz
        dd  filesize                ;  p_memsz
        dd  5                       ;  p_flags
        dd  0x1000                  ;  p_align
phdrsize  equ $ - phdr

_start:
  mov     eax, 1  ; "exit" syscall, see unistd.h
  mov     ebx, 42
  int     0x80
filesize      equ     $ - $$
\end{lstlisting}
\end{frame}

\begin{frame}[fragile]
\frametitle{Time for some black magic: let's write ELF directly.}
\begin{lstlisting}
$ nasm -f bin -o a.out tiny.asm
$ chmod +x a.out
$ ./a.out ; echo $?
42
$ wc -c a.out
     91 a.out
\end{lstlisting}
91 Bytes, not bad!

But the ELF header still contains to many unused bytes.
\end{frame}

\begin{frame}[fragile]
\frametitle{Wait\ldots the spec doesn't forbid \emph{overlapping} headers\ldots}
\begin{lstlisting}[basicstyle=\tiny\ttfamily]
; tiny.asm
BITS 32
         org     0x00200000
         db      0x7F, "ELF"     ; e_ident
         db      1, 1, 1, 0, 0
_start:  mov     bl, 42          ;  (padding)
         xor     eax, eax        ;  (no wait)
         inc     eax             ;  (what?)
         int     0x80            ;  (ohhh, cunning plan!)
         dw      2               ; e_type
         dw      3               ; e_machine
         dd      1               ; e_version
         dd      _start          ; e_entry
         dd      phdr - $$       ; e_phoff
phdr:    dd      1               ; e_shoff       ; p_type
         dd      0               ; e_flags       ; p_offset
         dd      $$              ; e_ehsize      ; p_vaddr
                                 ; e_phentsize
         dw      1               ; e_phnum       ; p_paddr
         dw      0               ; e_shentsize
         dd      filesize        ; e_shnum       ; p_filesz
                                 ; e_shstrndx
         dd      filesize                        ; p_memsz
         dd      5                               ; p_flags
         dd      0x1000                          ; p_align
filesize equ     $ - $$
\end{lstlisting}
\end{frame}

\begin{frame}[fragile]
\frametitle{Wait\ldots the spec doesn't forbid \emph{overlapping} headers\ldots}
\ldots other dirty hacks like eliminating bytes that are not read by the loader
anyway\ldots
\begin{lstlisting}
$ nasm -f bin -o a.out tiny.asm
$ chmod +x a.out
$ ./a.out ; echo $?
42
$ wc -c a.out
     45 a.out
\end{lstlisting}

45 bytes for a valid Linux executable?! \textbackslash{}o/

(okay, ``valid''\ldots probably only works on Linux, but hey, it works!)
\end{frame}

\begin{frame}[fragile]
\frametitle{So, can we still do better?}
Unfortunately not.

There is no way to eliminate the last byte at file offset 45, which specifies
the location of the program header. This byte must be at position 45, and there
is no way around it.

On the other hand, we started out with 4.3~Kilobyte, and now we have 45~Byte.
That is quite an achievement.
\end{frame}

\begin{frame}{Sources}
\hfill
\small
\begin{thebibliography}{9}
	\bibitem{raiter}{Brian Raiter: A Whirlwind Tutorial on Creating Really Teensy
		ELF Executables for Linux. July 2, 1999.
		\url{http://www.muppetlabs.com/~breadbox/software/tiny/teensy.html}}
\end{thebibliography}

\hfill

This presentation was carefully copy-pasted from there by hand. No code segments
were harmed in the making of this presentation.

(And since the original post did not specify a licence, I am probably doomed
now. Brian may forgive me, but his write-up is just so excellent.)
\hfill
\end{frame}
\end{document}

